%\usepackage[display]{texpower}

% CHANGED: Added
\documentclass[slidestop,usepdftitle=false, xcolor=table]{beamer}
\usepackage[accumulated]{beamerseminar}
\usepackage[spanish]{babel}
\usepackage[utf8]{inputenc}
\usepackage{beamertexpower}
\usepackage{multicol}
\usepackage{beamerthemeshadow}
\usepackage{graphics}
\usepackage{graphicx}
\usepackage{color}
\usepackage{animate}
\usepackage{subfigure}
\usepackage{movie15}
\usepackage{bibentry}
\usepackage{epstopdf}
\usepackage{lmodern}
%\usepackage{enumerate}
\usepackage{amssymb,amsmath,graphicx,tikz,movie15}
\usepackage[linesnumbered,lined,boxed,commentsnumbered,ruled]{algorithm2e}
\usepackage{hyperref}
\usepackage[mathscr]{eucal}
\usepackage{optidef}
\setbeamertemplate{theorems}[numbered]
\setbeamercolor{frametitle}{bg=black}

\usetheme{Antibes}

\definecolor{studentbrown}{RGB}{124,71,50}

\newtheorem{act}{Actividad}

\newtheorem{ej}{Ejemplo}[section]
% Theorem %%%%%%%%%%%%%%%%%%%%%%%%%%%%%%%%%%%%%%%%%%%%%%%%%%%%%%%%%%%%
\BeforeBeginEnvironment{ej}{
	\setbeamercolor{block title}{use=example text,fg=white,bg=example text.fg!75!black}
	\setbeamercolor{block body}{parent=normal text,use=block title example,bg=block title example.bg!10!bg}
}
\AfterEndEnvironment{ej}{
	\setbeamercolor{block title}{use=structure,fg=white,bg=structure.fg!75!black}
	\setbeamercolor{block body}{parent=normal text,use=block title,bg=block title.bg!10!bg}
}

% Proposition %%%%%%%%%%%%%%%%%%%%%%%%%%%%%%%%%%%%%%%%%%%%%%%%%%%%%%%%
\newtheorem{obs}{Observación}[section]
\BeforeBeginEnvironment{obs}{
	\setbeamercolor{block title}{use=alerted text,fg=white,bg=alerted text.fg!75!black}
	\setbeamercolor{block body}{parent=normal text,use=block title alerted,bg=block title alerted.bg!10!bg}
}
\AfterEndEnvironment{obs}{
	\setbeamercolor{block title}{use=structure,fg=white,bg=structure.fg!75!black}
	\setbeamercolor{block body}{parent=normal text,use=block title,bg=block title.bg!10!bg}
}

% Definition %%%%%%%%%%%%%%%%%%%%%%%%%%%%%%%%%%%%%%%%%%%%%%%%%%%%%%%%
\newtheorem{defi}{Definición}[section]
\BeforeBeginEnvironment{defi}{
	\setbeamercolor{block title}{use=alerted text,fg=white,bg=alerted text.fg!75!black}
	\setbeamercolor{block body}{parent=normal text,use=block title alerted,bg=block title alerted.bg!10!bg}
}
\AfterEndEnvironment{defi}{
	\setbeamercolor{block title}{use=structure,fg=white,bg=structure.fg!75!black}
	\setbeamercolor{block body}{parent=normal text,use=block title,bg=block title.bg!10!bg}
}

%%%%%%%%%%%%%%%%%
\definecolor{31}{rgb}{.3,.5,.2}
\definecolor{13}{rgb}{.1,.6,.3}
% CHANGED: Moved \title and \author outside of slide
\title[Problemas de coordinación de drones con otros vehículos]{Problemas de coordinación de drones con otros vehículos}
\author{Lavinia Amorosi, Justo Puerto y Carlos Valverde}
%\newline
\institute{\begin{center}
		\date{2020/2021}
		\text{Instituto de Matemáticas de la Universidad de Sevilla}
	\end{center}
	% \begin{center}
	% 	\textcolor{blue}{Lavinia Amorosi, Justo Puerto y Carlos Valverde}
	% \end{center}
	\begin{center}
		\includegraphics[width=0.3\linewidth]{Logo_IMUS}
	\end{center}

}

\begin{document}

	\begin{frame}
		\titlepage
	\end{frame}

	\begin{frame}
		\frametitle{Indice de contenidos}
		\begin{itemize}
			\item Introducción
			\item Revisión de la literatura
			\item Descripción de los problemas
			\item Nuevos horizontes
		\end{itemize}
	\end{frame}

	\setbeamerfont{frametitle}{size=\large}
	\section{Introducción}
	\begin{frame}
		\begin{center}
			\includegraphics[width=0.85\linewidth]{map_mathematics}
		\end{center}
		PD: Gracias a LA REME por su poder inspirador.
	\end{frame}

	\begin{frame}
		\frametitle{Uso de los drones}
		Se está empleando en multitud de campos (Otto, 2018):
		\begin{itemize}
			\item \textbf{Telecomunicaciones}, proporcionando conectividad a la red en áreas rurales.
			\pause
			\item \textbf{Catástrofes naturales}, cuando la infraestructura del lugar está afectada.
			\pause
			\item \textbf{Actividades de inspección}.
			\pause
			\item \textbf{Reparto de bienes}.
			\begin{center}
				\includegraphics[width=0.55\linewidth]{imagen_dron}
			\end{center}
		\end{itemize}
	\end{frame}

	\begin{frame}
		\frametitle{Ventajas con respecto a transportes tradicionales}
		\begin{itemize}
			\item Costes de transporte menores.
			\pause
			\item Tecnología más limpia.
			\pause
			\item Reemplaza la actividad humana en algunas actividades de riesgo.
			\pause
			\item Accesibilidad a cualquier zona. (\url{https://www.youtube.com/watch?v=e7x-Z54ipug})
			\item \textbf{Libertad de movimiento}.
		\end{itemize}
		\begin{center}
			\includegraphics[width=0.55\linewidth]{dron_inspeccion}
		\end{center}
	\end{frame}

	\begin{frame}
		\frametitle{Inconveniente con respecto a transportes tradicionales}
		El mayor inconveniente que presenta esta tecnología es la \textbf{falta de autonomía} para completar un servicio, inconveniente que se resuelve coordinando drones con naves tradicionales que hacen de soporte.
		\begin{center}
			\includegraphics[width=0.8\linewidth]{dron_camion}
		\end{center}
	\end{frame}

	\section{Revisión de la literatura}
	\begin{frame}
		\frametitle{Un nuevo campo en el diseño de rutas}
		\begin{itemize}
		\item La mayoría de los problemas clásicos de rutas que se han estudiado, se resuelven sobre grafos. Por ejemplo, el clásico Traveling Salesman Problem:
		\begin{center}
			\includegraphics[width=0.4\linewidth]{TSP}
		\end{center}
		\pause
		\item Sin embargo, como los drones se pueden mover \textit{libremente} por el espacio, se tiene la libertad de que puede realizar cualquier trayectoria, abriendo una nueva línea de investigación en el diseño de rutas.
		\end{itemize}
	\end{frame}

	\begin{frame}
		\frametitle{Punto de partida de nuestro trabajo}
		En 2018, Stefan Poikonen y Bruce Golden definieron (aunque no formularon explícitamente) The Mothership and Drone Routing Problem (MDRP):
		\begin{center}
			\includegraphics[width=0.5\linewidth]{poikonen_2}
		\end{center}
	\end{frame}

	\begin{frame}
		\frametitle{Punto de partida de nuestro trabajo}
		En 2018, Stefan Poikonen y Bruce Golden definieron (aunque no formularon explícitamente) The Mothership and Drone Routing Problem (MDRP):
		\begin{center}
			\includegraphics[width=0.5\linewidth]{poikonen_1}
		\end{center}
	\end{frame}

	\begin{frame}
		\textbf{Observación:} Nuestros modelos tienen más apellidos que La duquesa de Alba de Tormes:
		\begin{center}
			\includegraphics[width=0.4\linewidth]{duquesa2}\includegraphics[width=0.4\linewidth]{apellidos_duquesa}
		\end{center}
	\end{frame}

	\section{Descripción del problema}
	\begin{frame}
		\frametitle{Extensiones del MDRP}
		¿Cómo podemos extender este problema en cuanto al movimiento del dron?
		\begin{enumerate}
			\item Formulamos explícitamente el problema.
			\pause
			\item Cambiamos los objetivos a visitar de puntos a poligonales que se tenían que atravesar en un porcentaje.
			\pause
			\item Cambiamos los objetivos a visitar de poligonales a grafos visitando cada arista a lo sumo una vez y recorriéndolos \ldots
			\begin{itemize}
				\item en un porcentaje de su longitud total.
				\item en un porcentaje de la longitud de cada una de sus aristas.
			\end{itemize}
		\end{enumerate}
		A esto lo hemos llamado \textbf{The All-Terrain Mothership and Drone Routing Problem with Graphs} (AMDRPG). Este problema puede simular el movimiento de un dron en coordinación con un medio aéreo o un medio marítimo.
	\end{frame}

	\begin{frame}
		\frametitle{Datos iniciales}
		\begin{center}
			\includegraphics[width=0.8\linewidth]{PDMTZ_1}
		\end{center}
	\end{frame}

	\begin{frame}
		\frametitle{Recorrido un porcentaje (50\%) de la longitud de cada arista del grafo}
		\begin{center}
			\includegraphics[width=0.8\linewidth]{PDMTZ_e}
		\end{center}
	\end{frame}

	\begin{frame}
		\frametitle{Recorrido un porcentaje (50\%) de la longitud total del grafo}
		\begin{center}
			\includegraphics[width=0.8\linewidth]{PDMTZ_g}
		\end{center}
	\end{frame}

	\begin{frame}
		\frametitle{Limitación del AMDRPG}
		La principal limitación que tiene este problema es el hecho de que los puntos de lanzamiento y recogida están localizados libremente, algo que no es realista con el movimiento de un vehículo.
		\begin{small}
		\begin{enumerate}
			\item Cambiamos el caso en el que la nave nodriza se movía libremente a aquel en el que se mueve sobre una poligonal cerrada, lo que hemos denominado como \textbf{The Network Mothership-Drone Routing Problem on a Polygonal with Graphs} (PMDRPG). (El primo hermano de la Duquesa)
			\begin{center}
				\includegraphics[width=0.5\linewidth]{TDMTZ}
			\end{center}
		\end{enumerate}
	\end{small}
	\end{frame}

	\begin{frame}
		\frametitle{Limitación del AMDRPG}
		La principal limitación que tiene este problema es el hecho de que los puntos de lanzamiento y recogida están localizados libremente, algo que no es realista con el movimiento de un vehículo.
		\begin{small}
			\begin{enumerate}
				\setcounter{enumi}{1}
				\item Dimos un paso más allá, presentando una formulación en la que el camión se mueve sobre un grafo, cuyo nombre es \textbf{The Network Mothership-Drone Routing Problem with Graphs} (NMDRPG). (La hermana del primo de la Duquesa)
				\begin{center}
					\includegraphics[width=0.5\linewidth]{NDMTZ}
				\end{center}
			\end{enumerate}
		\end{small}
	\end{frame}

	\begin{frame}
		\frametitle{¿Qué dificultades presenta este tipo de problemas?}
		\begin{itemize}
			\item Al tener que diseñar dos rutas coordinadas, el problema se vuelve combinatorio \textit{de forma cuadrática}.
			\item Generan una gran cantidad de variables y restricciones con una cantidad de datos relativamente pequeña. Lo que dificulta la búsqueda del óptimo.
		\end{itemize}
		\begin{center}
			\includegraphics[width=0.25\linewidth]{cerebro}
		\end{center}
	\end{frame}

	\begin{frame}
		\frametitle{Un algoritmo matheurístico para este tipo de problemas}
		Al ser problemas \textit{muy} combinatorios, lo que se suele buscar son algoritmos que proporcionen \textit{buenas soluciones en un tiempo razonable}. La idea de nuestro algoritmo es descomponer el problema en dos subproblemas fijando variables.
		% \begin{block}
		\begin{footnotesize}
			\begin{enumerate}
				\item Buscamos los centroides de los vértices que forman cada grafo. Y generamos un círculo de un radio determinado centrado en dichos centroides.
				\item Resolvemos el Traveling Salesman Problem with Neighborhoods (TSPN) sobre estos círculos para determinar el orden general de recorrido de los grafos.
				\item Una vez fijado este orden, buscamos el punto de lanzamiento y recogida de forma recursiva sobre cada grafo para fijar el orden de visita dentro de cada grafo.
				\item Resolvemos el modelo con todos estos puntos fijados pero dejando libre el orden general de recorrido.
				\item Vemos si este orden coincide con el anterior. Si no, se repite el Paso 4.
			\end{enumerate}
		\end{footnotesize}
		% \end{block}
	\end{frame}

	\begin{frame}
		\frametitle{Un algoritmo matheurístico para este tipo de problemas}
			\begin{figure}%
				\hspace{-2.5cm}
		    \includegraphics[width=7cm]{step0}\hspace{-1.5cm}\includegraphics[width=7cm]{step1}%
		    \caption{Ejemplo ilustrativo}%
		    \label{fig:example}%
			\end{figure}
	\end{frame}

	\section{Nuevos horizontes}
	\begin{frame}
		\frametitle{Nuevos horizontes}
		Una vez formulados estos problemas, queremos profundizar en este tema en un siguiente documento que pueda responder a las siguientes preguntas en orden de dificultad:
		\begin{itemize}
			\item ¿Qué ocurre si, en vez de disponer solo un dron, tenemos varios drones que podemos lanzar desde un mismo punto y recogerlos a todos en otro punto?
			\pause
			\item ¿Qué ocurre si el dron tiene la suficiente capacidad para visitar más de un elemento antes de regresar a la nave nodriza?
			\pause
			\item ¿Es posible formular el problema en el caso en el que el movimiento de la nave nodriza y/o el dron presente barreras que no puede atravesar?
		\end{itemize}
	\end{frame}

	\begin{frame}
		\frametitle{Nuevos horizontes}
		\begin{center}
			\includegraphics[width=0.5\linewidth]{continuara}
		\end{center}
	\end{frame}

	\begin{frame}
		\vspace{1.5cm}
		\hspace{1.5cm}
		\includegraphics[width=0.7\linewidth]{gracias}
	\end{frame}








\end{document}
