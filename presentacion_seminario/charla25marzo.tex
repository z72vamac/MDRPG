\documentclass[slidestop,usepdftitle=false, xcolor=table]{beamer}
\usepackage[accumulated]{beamerseminar}
\usepackage[spanish]{babel}
\usepackage[utf8]{inputenc}
\usepackage{beamertexpower}
\usepackage{multicol}
\usepackage{beamerthemeshadow}
\usepackage{graphics}
\usepackage{graphicx}
\usepackage{color}
\usepackage{animate}
\usepackage{subfigure}
\usepackage{movie15}
\usepackage{bibentry}
\usepackage{epstopdf}
\usepackage{lmodern}
%\usepackage{enumerate}
\usepackage{amssymb,amsmath,graphicx,tikz,movie15}
\usepackage[linesnumbered,lined,boxed,commentsnumbered,ruled]{algorithm2e}
\usepackage{hyperref}
\usepackage[mathscr]{eucal}
\usepackage{optidef}
\setbeamertemplate{theorems}[numbered]
\setbeamercolor{frametitle}{bg=black}

\usetheme{Antibes}

\definecolor{studentbrown}{RGB}{124,71,50}

\newtheorem{act}{Actividad}

\newtheorem{ej}{Ejemplo}[section]
% Theorem %%%%%%%%%%%%%%%%%%%%%%%%%%%%%%%%%%%%%%%%%%%%%%%%%%%%%%%%%%%%
\BeforeBeginEnvironment{ej}{
	\setbeamercolor{block title}{use=example text,fg=white,bg=example text.fg!75!black}
	\setbeamercolor{block body}{parent=normal text,use=block title example,bg=block title example.bg!10!bg}
}
\AfterEndEnvironment{ej}{
	\setbeamercolor{block title}{use=structure,fg=white,bg=structure.fg!75!black}
	\setbeamercolor{block body}{parent=normal text,use=block title,bg=block title.bg!10!bg}
}

% Proposition %%%%%%%%%%%%%%%%%%%%%%%%%%%%%%%%%%%%%%%%%%%%%%%%%%%%%%%%
\newtheorem{obs}{Observación}[section]
\BeforeBeginEnvironment{obs}{
	\setbeamercolor{block title}{use=alerted text,fg=white,bg=alerted text.fg!75!black}
	\setbeamercolor{block body}{parent=normal text,use=block title alerted,bg=block title alerted.bg!10!bg}
}
\AfterEndEnvironment{obs}{
	\setbeamercolor{block title}{use=structure,fg=white,bg=structure.fg!75!black}
	\setbeamercolor{block body}{parent=normal text,use=block title,bg=block title.bg!10!bg}
}

% Definition %%%%%%%%%%%%%%%%%%%%%%%%%%%%%%%%%%%%%%%%%%%%%%%%%%%%%%%%
\newtheorem{defi}{Definición}[section]
\BeforeBeginEnvironment{defi}{
	\setbeamercolor{block title}{use=alerted text,fg=white,bg=alerted text.fg!75!black}
	\setbeamercolor{block body}{parent=normal text,use=block title alerted,bg=block title alerted.bg!10!bg}
}
\AfterEndEnvironment{defi}{
	\setbeamercolor{block title}{use=structure,fg=white,bg=structure.fg!75!black}
	\setbeamercolor{block body}{parent=normal text,use=block title,bg=block title.bg!10!bg}
}

%%%%%%%%%%%%%%%%%
\definecolor{31}{rgb}{.3,.5,.2}
\definecolor{13}{rgb}{.1,.6,.3}
% CHANGED: Moved \title and \author outside of slide
\title[Problemas de coordinación de drones con otros vehículos]{Problemas de coordinación de drones con otros vehículos}
\author{Lavinia Amorosi, Justo Puerto y Carlos Valverde}
%\newline
\institute{\begin{center}
		\date{2020/2021}
		\text{Instituto de Matemáticas de la Universidad de Sevilla}
	\end{center}
	% \begin{center}
	% 	\textcolor{blue}{Lavinia Amorosi, Justo Puerto y Carlos Valverde}
	% \end{center}
	\begin{center}
		\includegraphics[width=0.3\linewidth]{Logo_IMUS}
	\end{center}

}

\begin{document}

	\begin{frame}
		\titlepage
	\end{frame}

	\begin{frame}
		\frametitle{Indice de contenidos}
		\begin{itemize}
			\item Introducción
			\item Revisión de la literatura
			\item Descripción de los problemas
			\item Nuevos horizontes
		\end{itemize}
	\end{frame}

%	\setbeamerfont{frametitle}{size=\large}
%	\section{Introducción}
%	\begin{frame}
%		\begin{center}
%			\includegraphics[width=0.85\linewidth]{map_mathematics}
%		\end{center}
%		PD: Gracias a LA REME por su poder inspirador.
%	\end{frame}

	\begin{frame}
		\frametitle{Uso de los drones}
		Se está empleando en multitud de campos (Otto, 2018):
		\begin{itemize}
			\item \textbf{Telecomunicaciones}, proporcionando conectividad a la red en áreas rurales.
			\pause
			\item \textbf{Catástrofes naturales}, cuando la infraestructura del lugar está afectada.
			\pause
			\item \textbf{Actividades de inspección}.
			\pause
			\item \textbf{Reparto de bienes}.
			\begin{center}
				\includegraphics[width=0.55\linewidth]{imagen_dron}
			\end{center}
		\end{itemize}
	\end{frame}

	\begin{frame}
		\frametitle{Ventajas con respecto a transportes tradicionales}
		\begin{itemize}
			\item Costes de transporte menores.
			\pause
			\item Tecnología más limpia.
			\pause
			\item Reemplaza la actividad humana en algunas actividades de riesgo.
			\pause
			\item Accesibilidad a cualquier zona. (\url{https://www.youtube.com/watch?v=e7x-Z54ipug})
			\item \textbf{Libertad de movimiento}.
		\end{itemize}
		\begin{center}
			\includegraphics[width=0.55\linewidth]{dron_inspeccion}
		\end{center}
	\end{frame}

	\begin{frame}
		\frametitle{Inconveniente con respecto a transportes tradicionales}
		El mayor inconveniente que presenta esta tecnología es la \textbf{falta de autonomía} para completar un servicio, inconveniente que se resuelve coordinando drones con naves tradicionales que hacen de soporte.
		\begin{center}
			\includegraphics[width=0.8\linewidth]{dron_camion}
		\end{center}
	\end{frame}

	\section{Revisión de la literatura}
	\begin{frame}
		\frametitle{Un nuevo campo en el diseño de rutas}
		\begin{itemize}
		\item La mayoría de los problemas clásicos de rutas que se han estudiado, se resuelven sobre grafos. Por ejemplo, el clásico Traveling Salesman Problem:
		\begin{center}
			\includegraphics[width=0.4\linewidth]{TSP}
		\end{center}
		\pause
		\item Sin embargo, como los drones se pueden mover \textit{libremente} por el espacio, se tiene la libertad de que puede realizar cualquier trayectoria, abriendo una nueva línea de investigación en el diseño de rutas.
		\end{itemize}
	\end{frame}

	\begin{frame}
		\frametitle{Punto de partida de nuestro trabajo}
		En 2018, Stefan Poikonen y Bruce Golden definieron (aunque no formularon explícitamente) The Mothership and Drone Routing Problem (MDRP):
		\begin{center}
			\includegraphics[width=0.5\linewidth]{poikonen_2}
		\end{center}
	\end{frame}

	\begin{frame}
		\frametitle{Punto de partida de nuestro trabajo}
		En 2018, Stefan Poikonen y Bruce Golden definieron (aunque no formularon explícitamente) The Mothership and Drone Routing Problem (MDRP):
		\begin{center}
			\includegraphics[width=0.5\linewidth]{poikonen_1}
		\end{center}
	\end{frame}

%	\begin{frame}
%		\textbf{Observación:} Nuestros modelos tienen más apellidos que La duquesa de Alba de Tormes:
%		\begin{center}
%			\includegraphics[width=0.4\linewidth]{duquesa2}\includegraphics[width=0.4\linewidth]{apellidos_duquesa}
%		\end{center}
%	\end{frame}

	\section{Descripción del problema}
	\begin{frame}
		\frametitle{Extensiones del MDRP}
		¿Cómo podemos extender este problema en cuanto al movimiento del dron?
		\begin{enumerate}
			\item Formulamos explícitamente el problema.
			\pause
			\item Cambiamos los objetivos a visitar de puntos a poligonales que se tenían que atravesar en un porcentaje.
			\pause
			\item Cambiamos los objetivos a visitar de poligonales a grafos visitando cada arista a lo sumo una vez y recorriéndolos \ldots
			\begin{itemize}
				\item en un porcentaje de su longitud total.
				\item en un porcentaje de la longitud de cada una de sus aristas.
			\end{itemize}
		\end{enumerate}
		A esto lo hemos llamado \textbf{The All-Terrain Mothership and Drone Routing Problem with Graphs} (AMDRPG). Este problema puede simular el movimiento de un dron en coordinación con un medio aéreo o un medio marítimo.
	\end{frame}

	\begin{frame}
		\frametitle{Datos iniciales}
		\begin{itemize}
		\item $\mathcal G$: conjunto de grafos que se deben visitar.
		\item $orig/dest$: puntos de origen y destino.
		\item $\alpha^g$/$\alpha^{e_g}$: porcentaje de recorrida de la longitud total del grafo o de cada arista.
		\item $v_D$ y $v_C$: velocidad del dron y de la nave nodriza, respectivamente.
		\end{itemize}
		\begin{center}
			\includegraphics[width=0.5\linewidth]{PDMTZ_1}
		\end{center}
	\end{frame}

	\begin{frame}
		\frametitle{Datos iniciales}
		Para cada grafo $g\in\mathcal G$ se denota por:
		\begin{itemize}
			\item $V_g$: conjunto de vértices de $g$.
			\item $E_g$: conjunto de aristas de $g$. Cada arista $e_g\in E_g$:
			\begin{itemize}
				\item Se parametriza por los puntos: $B^{e_g}$, $C^{e_g}$.
				\item Tiene como longitud: $\mathcal L(e_g)=\|\overline{B^{e_g}C^{e_g}}\|$.
				\item Tiene asociada una variable binaria $\mu^{e_g}$ que indica si se visita la arista o no.
				\item Tiene asignada un punto de entrada: $(R^{e_g}, \rho^{e_g})$.
				\item Tiene asignada un punto de salida: $(L^{e_g}, \lambda^{e_g})$.
			\end{itemize}
				
			\item $\mathcal L(g)$: longitud total de $g$.
		\end{itemize}
	\end{frame}
	
	\begin{frame}
		\frametitle{Tipos de recorrida: Visitando un porcentaje de cada arista}
		La siguiente restricción modela la visita de un porcentaje $\alpha^{e_g}$ de la longitud de cada una de sus aristas:
		\begin{equation}\label{eq:alpha-E}\tag{$\alpha$-E}
 \mu^{e_g}|\rho^{e_g}-\lambda^{e_g}|\geq \alpha^{e_g} \Longleftrightarrow
 \left\{
 \begin{array}{ccl}
  \rho^{e_g} - \lambda^{e_g}                       & =    & \nu_\text{max}^{e_g} - \nu_\text{min}^{e_g}                                     \\
  \nu_\text{max}^{e_g}                         & \leq & 1-{\text{entry}^{e_g}}                                    \\
  \nu_\text{min}^{e_g}                      & \leq & {  \text{entry}^{e_g}},                                        \\
  \mu^{e_g}(\nu_\text{max}^{e_g} + \nu_\text{min}^{e_g} ) & \geq & \alpha^{e_g}
  \\
 \end{array}
 \right.
\end{equation}
donde $\text{entry}^{e_g}$ determina el sentido del viaje en $e_g$ y $\nu_\text{min}^{e_g}$ y $\nu_\text{max}^{e_g}$ son variables auxiliares que permiten linearizar el valor absoluto. 
\end{frame}

	\begin{frame}
		\frametitle{Tipos de recorrida: Visitando un porcentaje del grafo}
La siguiente restricción modela la visita de un porcentaje $\alpha^g$ de la longitud total del grafo:		

\begin{equation}\label{eq:alphaG}\tag{$\alpha$-G}
    \sum_{e_g\in E_g} \mu^{e_g}|\lambda^{e_g} - \rho^{e_g}|\mathcal L(e_g) \geq \alpha^g\mathcal L(g),
\end{equation}
La linearización de \eqref{eq:alphaG} es similar a la de \eqref{eq:alpha-E} cambiando la última restricción por:
\begin{equation}\label{eq:alpha-G}\tag{$\alpha$-G}
\sum_{e_g\in E_g} \mu^{e_g}(\nu_\text{max}^{e_g} + \nu_\text{min}^{e_g})\mathcal L(e_g)\geq \alpha^g\mathcal L(g).
\end{equation}
\end{frame}

\section{Formulación del (AMDRPG) basado en etapas}

\begin{frame}
	\frametitle{Variables binarias del (AMDRPG) basado en etapas}
	\begin{itemize}
	\item En esta formulación, se identifica cada grafo a una etapa $t\in T:=\{1,\ldots,|\mathcal G|\}$ y su respectivos puntos de lanzamiento $x_L^t$ y recogida $x_R^t$.
	\item Para cada etapa $t$ el camino que realiza el dron es el siguiente:
	 \begin{equation*}
	 x_L^t\rightarrow R^{e_g}\rightarrow\ldots\rightarrow R^{e^\prime_g}\rightarrow L^{e^\prime_g}\rightarrow \ldots \rightarrow R^{e''_g} \rightarrow x_R^t\rightarrow x_L^{t+1}.
	 \end{equation*}
	 \item Esto sugiere definir las siguientes variables binarias:
	 \begin{itemize}
	 	\item $u^{e_gt} = 1$ si el dron entra al grafo $g$ por el segmento $e_g$ en la etapa $t$.
   		\item $z^{e_ge^\prime_g} = 1$ si el dron va de la arista $e_g$ a la arista $e'_g$.
    	\item $v^{e_gt} = 1$ si el dron sale del grafo $g$ por la arista $e_g$ en la etapa $t$.
	\end{itemize}
	\end{itemize}
	\end{frame}
\begin{frame}
	\frametitle{Variables continuas del (AMDRPG) basado en etapas}
	
\begin{footnotesize}
	Por otro lado, se definen también las variables continuas relacionadas con la distancia realizadas por el dron y la nave nodriza:
	\begin{itemize}
    \item $d_L^{e_gt} = \|x_L^t - R^{e_g}\|$. Distancia realizada por el dron desde el punto de lanzamiento en la etapa $t$ al primer punto que se visita en el grafo $g$. 
    \item $d^{e_ge^\prime_g} = \|R^{e_g} - L^{e^\prime_g}\|$. Distancia realizada por el dron desde el punto de lanzamiento en $e_g$ al punto de recogida en $e'_g$.
    \item $d^{e_g} = \|R^{e_g} - L^{e_g}\|$. Distancia realizada por el dron desde el punto de recogida en $e_g$ al punto de lanzamiento en esa arista $e_g$.
    \item $d_R^{e_gt} = \|L^{e_g} - x_R^t\|$. Distancia realizada por el dron desde el punto de lanzamiento de $e_g$ al punto de recogida en el segmento $e_g$.
    \item $d_{LR}^t = \|x_L^t - x_R^t\|$. Distancia realizada por la nave nodriza desde el punto de lanzamiento $x_L^t$ al punto de recogida $x_R^t$ asociada a la etapa $t$.
    \item $d_{RL}^t = \|x_R^t - x_L^{t+1}\|$. Distancia realizada por la nave nodriza desde el punto de recogida $x_R^t$ al punto de lanzamiento asociado a la siguiente etapa $x_L^{t+1}$.
\end{itemize}
\end{footnotesize}

\end{frame}

	\begin{frame}
		\frametitle{Restricciones del (AMDRPG) basado en etapas}
		Usando las variables binarias anteriores, podemos modelar la ruta que sigue el dron:
		\begin{align}
    \sum_{g\in \mathcal G}\sum_{e_g\in E_g} u^{e_gt} & = 1, &\forall t\in T \label{st:DEnt}\\%\tag{DEn}\\
    \sum_{g\in\mathcal G}\sum_{e_g\in E_g} v^{e_gt} & = 1, &\forall t\in T \label{st:DExt}\\%\tag{DEx}\\
    \sum_{e_g\in E_g} \sum_{t\in T} u^{e_gt} & = 1, &\forall g\in\mathcal G \label{st:DEng}\\%\tag{D
    \sum_{e_g\in E_g}\sum_{t\in T} v^{e_gt} & = 1, &\forall g\in\mathcal G \label{st:DExg}%\tag{D
\end{align}

	\end{frame}

	\begin{frame}
		\frametitle{Restricciones del (AMDRPG) basado en etapas}
		Usando las variables binarias anteriores, podemos modelar la ruta que sigue el dron:
		\begin{align}
    \sum_{e_g\in E_g} u^{e_gt} & = \sum_{e_g\in E_g} v^{e_gt}, &\forall g\in\mathcal G, \forall t\in T \label{st:Duv}\\%\tag{D
    \sum_{e^\prime_g\in E_g} z^{e^\prime_ge_g} + \sum_{t\in T} u^{e_gt} & = \mu^{e_g}, &\forall e_g\in E_g:g\in\mathcal G \label{st:DInu}\\
    \sum_{e^\prime_g\in E_g} z^{e_ge^\prime_g} + \sum_{t\in T} v^{e_gt} & = \mu^{e_g}, &\forall e_g\in E_g:g\in\mathcal G \label{st:DInv}
\end{align}

	\end{frame}
	
	\begin{frame}
		\frametitle{Restricciones del (AMDRPG) basado en etapas}
		\begin{small}
		Para asegurar que no existen subtours a la hora de recorrer las aristas del grafo: 
		\begin{itemize}
		\item Se pueden añadir las desigualdades de Miller-Tucker-Zemlin, dadas por:
		\begin{align}
		    s^{e_g} - s^{e^\prime_g} + |E_g|z^{e_ge^\prime_g} & \leq |E_g| - 1  , &\quad\forall e_g \neq e_g'\in E_g \tag{MTZ$_1$}, \label{MTZ1}\\
    0 & \leq s^{e_g} \leq |E_g| - 1 &\quad\forall e_g\in E_g\tag{MTZ$_2$},\label{MTZ2}
    \end{align}
    donde $s^{e_g}$ son variables continuas que indican el orden de visita de las aristas del grafo.
    \pause
    \item También se puede incluir la familia exponencial de restricciones de eliminación de subtours:
    \begin{equation}\tag{SEC}\label{SEC}
    \sum_{e_g, e^\prime_g \in S} z_g^{e_ge^\prime_g} \leq |S| - 1, \quad \forall S\subset E_g:g\in \mathcal G
\end{equation}
\end{itemize}
\end{small}
\end{frame}


\begin{frame}
	\frametitle{Restricciones del (AMDRPG) basado en etapas}
	Para coordinar la nave nodriza con el dron, se define la siguiente restricción para cada $t\in T$ y $g\in\mathcal G$:
	\begin{tiny}
	\begin{equation}\tag{DCW-t}\label{DCW-t}
\left(\sum_{e_g\in E_g} u^{e_gt}d_L^{e_gt} + \sum_{e_g, e^\prime_g\in E_g}z^{e_ge^\prime_g}d^{e_ge^\prime_g} + \sum_{e_g\in E_g} \mu^{e_g}d^{e_g} + \sum_{e_g\in E_g} v^{e_gt}d_R^{e_gt}\right)/v_D \leq d_{RL}^t/v_M + M(1 - \sum_{e_g\in E_g} u^{e_gt})
\end{equation}	
	\end{tiny}

	
\end{frame}

\begin{frame}
	\frametitle{Formulación del (AMDRPG) basado en etapas}
	\begin{mini*}|s|
 {}{\sum_{g\in\mathcal G}\sum_{e_g\in E_g}\sum_{t\in T} (u^{e_gt}d_L^{e_gt} + v^{e_gt}d_R^{e_gt}) + \sum_{g\in\mathcal G}\sum_{e_g\in E_g} \mu^{e_g}d^{e_g} + }{}{}
 \breakObjective{+
 \sum_{g\in\mathcal G}\sum_{e_g,e^\prime_g\in E_g}z^{e_ge^\prime_g}d^{e_ge^\prime_g} + \sum_{t\in T} (d_{RL}^t + d_{LR}^t)} %\label{AMDRPG-ST} \tag{AMDRPG-ST}
 \addConstraint{\eqref{st:DEnt}-\eqref{st:DInv}}
 \addConstraint{\eqref{MTZ1} - \eqref{MTZ2} \text{ or } \eqref{SEC} }
 \addConstraint{\eqref{eq:alpha-E} \text{ or } \eqref{eq:alpha-G}}
 \addConstraint{\eqref{DCW-t}}
 	\end{mini*}
\end{frame}

\begin{frame}
	\frametitle{Formulación del (AMDRPG) basado en etapas}
	\begin{mini*}|s|
 {}{\sum_{g\in\mathcal G}\sum_{e_g\in E_g}\sum_{t\in T} (u^{e_gt}d_L^{e_gt} + v^{e_gt}d_R^{e_gt}) + \sum_{g\in\mathcal G}\sum_{e_g\in E_g} \mu^{e_g}d^{e_g} + }{}{}
 \breakObjective{+
 \sum_{g\in\mathcal G}\sum_{e_g,e^\prime_g\in E_g}z^{e_ge^\prime_g}d^{e_ge^\prime_g} + \sum_{t\in T} (d_{RL}^t + d_{LR}^t)} %\label{AMDRPG-ST} \tag{AMDRPG-ST}
 \addConstraint{\|x_L^t- R^{e_g}\|}{\leq d_L^{e_gt},}{\forall e_g:g\in\mathcal G, \forall t\in T}
 \addConstraint{\|R^{e_g}- L^{e_g}\|}{\leq d^{e_g},}{\forall e_g:g\in\mathcal G,\forall t\in T}
 \addConstraint{\|R^{e_g}- L^{e^\prime_g}\|}{\leq d^{e_ge^\prime_g},}{\forall e_g\neq e_g'\in E_g:g\in\mathcal G}
 \addConstraint{\|L^{e_g}- x_R^t\|}{\leq d_R^{e_gt},}{\forall e_g:g\in\mathcal G,\forall t\in T}
 \addConstraint{\|x_R^t- x_L^{t+1}\|}{\leq d_{RL}^t,}{\forall t\in T}
 \addConstraint{\|x_L^t- x_R^t\|}{\leq d_{LR}^t,}{\forall t\in T}
 	\end{mini*}
\end{frame}

\begin{frame}
	\frametitle{Formulación del (AMDRPG) basado en etapas}
	\begin{mini*}|s|
 {}{\sum_{g\in\mathcal G}\sum_{e_g\in E_g}\sum_{t\in T} (u^{e_gt}d_L^{e_gt} + v^{e_gt}d_R^{e_gt}) + \sum_{g\in\mathcal G}\sum_{e_g\in E_g} \mu^{e_g}d^{e_g} + }{}{}
 \breakObjective{+
 \sum_{g\in\mathcal G}\sum_{e_g,e^\prime_g\in E_g}z^{e_ge^\prime_g}d^{e_ge^\prime_g} + \sum_{t\in T} (d_{RL}^t + d_{LR}^t)} %\label{AMDRPG-ST} \tag{AMDRPG-ST}
 \addConstraint{x_L^0}{= orig}
 \addConstraint{x_R^0}{= orig}
 \addConstraint{x_L^{|\mathcal G|+1}}{= dest}
 \addConstraint{x_R^{|\mathcal G|+1}}{= dest.}
 	\end{mini*}
\end{frame}

		
\section{Formulación del (AMDRPG) forzando conectividad}
\begin{frame}
	En esta formulación, reemplazamos las variables $u^{\cdot t},\; v^{\cdot t}$ y las restricciones que modelan el tour usando etapas, (\ref{st:DEnt})-(\ref{st:DInv}), por restricciones que aseguran conectividad:
	\begin{itemize}
		\item Utilizando Miller-Tucker-Zemlin.
		\item Utilizando las SEC.
	\end{itemize}
\end{frame}

\begin{frame}
	\frametitle{Variables binarias del (AMDRPG) forzando conectividad}
	\begin{itemize}
	\item En esta formulación, se asocia un punto de lanzamiento ($x_L^g$) y recogida ($x_R^g$) a cada grafo $g\in\mathcal G$.
	\item Para cada grafo $g$ el camino que realiza el dron es el siguiente:
	 \begin{equation*}
	 x_L^g\rightarrow R^{e_g}\rightarrow\ldots\rightarrow R^{e^\prime_g}\rightarrow L^{e^\prime_g}\rightarrow \ldots \rightarrow R^{e''_g} \rightarrow x_R^g\rightarrow x_L^{g'}.
	 \end{equation*}
	 \item Esto sugiere definir las siguientes variables binarias:
\begin{itemize}
    \item $u^{e_g} = 1$ si el dron accede al grafo por el segmento $e_g$.
    \item $z^{e_ge^\prime_g} = 1$ si el dron va del segmento $e_g$ al segmento $e'_g$.
    \item $v^{e_g} = 1$ si el dron sale del grafo $g$ por el segmento $e_g$.
    \item \textcolor{red}{$w^{gg'} = 1$} si la nave nodriza va de $x_R^g$ a $x_L^{g'}$.
\end{itemize}
	\end{itemize}
	\end{frame}
\begin{frame}
	\frametitle{Variables continuas del (AMDRPG) forzando conectividad}
	
\begin{footnotesize}
	Por otro lado, se definen también las variables continuas relacionadas con la distancia realizadas por el dron y la nave nodriza:
\begin{itemize}
    \item $d_L^{e_g} = \|x_L^g - R^{e_g}\|$. Distancia realizada por el dron desde el punto de lanzamiento $x_L^g$ al primer punto que se visita en el segmento $e_g$.
    \item $d^{e_ge^\prime_g} = \|R^{e_g} - L^{e^\prime_g}\|$. Distancia realizada por el dron desde el punto de lanzamiento en $e_g$ al siguiente punto de recogida en $e'_g$.
    \item $d^{e_g} = \|R^{e_g} - L^{e_g}\|$. Distancia realizada por el dron dentro del segmento $e_g$.
    \item $d_R^{e_g} = \|L^{e_g} - x_R^g\|$. Distancia realizada por el dron desde el punto de lanzamiento del segmento $e_g$ al punto de recogida asociado al grafo $g$.
    \item $d_{LR}^g = \|x_L^g - x_R^g\|$. Distancia cubierta por la nave nodriza desde que lanza el dron hasta que lo recoge.
    \item $d_{RL}^{gg'} = \|x_R^g - x_L^{g'}\|$. Distancia hecha por la nave nodriza desde el punto de recogida asociado al grafo $g$ al punto de lanzamiento asociado al grafo $g'$.
\end{itemize}
\end{footnotesize}

\end{frame}

	\begin{frame}
		\frametitle{Restricciones del (AMDRPG) forzando conectividad}
		Usando las variables binarias anteriores, podemos modelar la ruta que sigue el dron:
\begin{align}
    \sum_{e_g\in E_g} u^{e_g} & = 1, \label{DEnt2}\\%\tag{DEn}\\
    \sum_{e_g\in E_g} v^{e_g} & = 1, \label{DExt}\\%\tag{DEx}\\
    \sum_{e^\prime_g\in E_g} z^{e^\prime_ge_g} + u^{e_g} & = \mu^{e_g}, &\quad\forall e_g\in E_g:g\in\mathcal G \label{DInu}\\
    \sum_{e^\prime_g\in E_g} z^{e_ge^\prime_g} + v^{e_g} & = \mu^{e_g}, &\quad\forall e_g\in E_g:g\in\mathcal G \label{DInv2}
    \end{align}

	\end{frame}

\begin{frame}
		\frametitle{Restricciones del (AMDRPG) forzando conectividad}
		Usando las variables binarias anteriores, podemos modelar la ruta que sigue la nave nodriza:
\begin{align}
    \sum_{g\in\mathcal G} w^{g0} & = 0, \label{TOrig}\\
    \sum_{g'\in\mathcal G} w^{(n_G+1) g'} & = 0, \label{TDest}\\
    \sum_{g'\in\mathcal G\setminus\{g\}} w^{gg'} & = 1, \label{TExt} &\quad\forall g\in \mathcal G\\
    \sum_{g\in\mathcal G\setminus\{g'\}} w^{gg'} & = 1, &\quad\forall g'\in \mathcal G\label{TEnt}\\
    s_g - s_{g'} + |\mathcal G|w^{gg'} & \leq |\mathcal G| - 1  , &\quad\forall g \neq g' \tag{MTZ$_3$} \label{MTZ3}\\
    0 & \leq s_g \leq |\mathcal G| - 1 &\quad\forall g\in \mathcal G\tag{MTZ$_4$}\label{MTZ4}\\
    s_0 & = 0, \tag{MTZ$_5$}\label{MTZ5}\\
    s_{n_G+1}&=n_G+1, \tag{MTZ$_6$}\label{MTZ6}
\end{align}

	\end{frame}


\begin{frame}
	\frametitle{Restricciones del (AMDRPG) forzando conectividad}
	Para coordinar la nave nodriza con el dron, se define la siguiente restricción para cada grafo $g\in\mathcal G$:
	\begin{tiny}
\begin{equation}
 \left(\sum_{e_g\in E_g} u^{e_g}d_L^{e_g} + \sum_{e_g, e^\prime_g\in E_g}z^{e_ge^\prime_g}d^{e_ge^\prime_g} + \sum_{e_g\in E_g} \mu^{e_g} d^{e_g} + \sum_{e_g\in E_g}v^{e_g}d_R^{e_g}\right)/v_D\leq d_{RL}^{g}/v_M, \quad\forall g\in\mathcal G.\label{DCW-g}\tag{DCW-g}
\end{equation}
	\end{tiny}

	
\end{frame}

\begin{frame}
	\frametitle{Formulación del (AMDRPG) forzando conectividad}
	\begin{mini*}|s|
{}{\sum_{g\in\mathcal G}\sum_{e_g\in E_g} (u^{e_g}d_L^{e_g} + v^{e_g}d_R^{e_g}) + \sum_{g\in\mathcal G}\sum_{e_g\in E_g} \mu^{e_g} d^{e_g} + }{}{}
\breakObjective{+\sum_{g\in\mathcal G}\sum_{e_g,e^\prime_g\in E_g}z^{e_ge^\prime_g}d^{e_ge^\prime_g} + \sum_{g\in\mathcal G} d_{LR}^g + \sum_{g,g'\in \mathcal G}d_{RL}^{gg'}w^{gg'}}%\label{AMDRPG-ST} \tag{AMDRPG-ST}
\addConstraint{\eqref{DEnt2}-\eqref{DInv2}}
 \addConstraint{\eqref{TOrig}-\eqref{TEnt}}
 \addConstraint{\eqref{MTZ1} - \eqref{MTZ2} \text{ or } \eqref{SEC} }
%\addConstraint{\eqref{MTZ3} - \eqref{MTZ6} \text{ or } \eqref{SEC} }{}{}
 \addConstraint{\eqref{MTZ3} - \eqref{MTZ6}}
 \addConstraint{\eqref{eq:alpha-E} \text{ or } \eqref{eq:alpha-G}}
 \addConstraint{\eqref{DCW-g}}
 	\end{mini*}
\end{frame}

\begin{frame}
	\frametitle{Formulación del (AMDRPG) forzando conectividad}
	\begin{mini*}|s|
{}{\sum_{g\in\mathcal G}\sum_{e_g\in E_g} (u^{e_g}d_L^{e_g} + v^{e_g}d_R^{e_g}) + \sum_{g\in\mathcal G}\sum_{e_g\in E_g} \mu^{e_g} d^{e_g} + }{}{}
\breakObjective{+\sum_{g\in\mathcal G}\sum_{e_g,e^\prime_g\in E_g}z^{e_ge^\prime_g}d^{e_ge^\prime_g} + \sum_{g\in\mathcal G} d_{LR}^g + \sum_{g,g'\in \mathcal G}d_{RL}^{gg'}w^{gg'}}  \addConstraint{\|x_L^g- R^{e_g}\|}{\leq d_L^{e_g},}{\forall e_g:g\in\mathcal G}
 \addConstraint{\|R^{e_g}- L^{e_g}\|}{\leq d^{e_g},}{\forall e_g:g\in\mathcal G}
 \addConstraint{\|R^{e_g}- L^{e^\prime_g}\|}{\leq d^{e_ge^\prime_g},}{\forall e_g\neq e'_g:g\in\mathcal G}
 \addConstraint{\|L^{e_g}- x_R^g\|}{\leq d_R^{e_g},}{\forall e_g:g\in\mathcal G}
 \addConstraint{\|x_R^g- x_L^{g'}\|}{\leq d^{gg'}_{RL},}{\forall g,g'\in\mathcal G}
 \addConstraint{\|x_L^g- x_R^g\|}{\leq d^g_{LR},}{\forall g\in\mathcal G}
 	\end{mini*}
\end{frame}

\begin{frame}
	\frametitle{Formulación del (AMDRPG) forzando conectividad}
	\begin{mini*}|s|
{}{\sum_{g\in\mathcal G}\sum_{e_g\in E_g} (u^{e_g}d_L^{e_g} + v^{e_g}d_R^{e_g}) + \sum_{g\in\mathcal G}\sum_{e_g\in E_g} \mu^{e_g} d^{e_g} + }{}{}
\breakObjective{+\sum_{g\in\mathcal G}\sum_{e_g,e^\prime_g\in E_g}z^{e_ge^\prime_g}d^{e_ge^\prime_g} + \sum_{g\in\mathcal G} d_{LR}^g + \sum_{g,g'\in \mathcal G}d_{RL}^{gg'}w^{gg'}}  \addConstraint{x_L^0}{= orig}{}
 \addConstraint{x_R^0}{= orig}{}
 \addConstraint{x_L^{|\mathcal G|+1}}{= dest}{}
 \addConstraint{x_R^{|\mathcal G|+1}}{= dest.}{}
 	\end{mini*}
\end{frame}

	

	\begin{frame}
		\frametitle{Recorrido un porcentaje (50\%) de la longitud de cada arista del grafo}
		\begin{center}
			\includegraphics[width=0.8\linewidth]{PDMTZ_e}
		\end{center}
	\end{frame}

	\begin{frame}
		\frametitle{Recorrido un porcentaje (50\%) de la longitud total del grafo}
		\begin{center}
			\includegraphics[width=0.8\linewidth]{PDMTZ_g}
		\end{center}
	\end{frame}

	\begin{frame}
		\frametitle{Limitación del AMDRPG}
		La principal limitación que tiene este problema es el hecho de que los puntos de lanzamiento y recogida están localizados libremente, algo que no es realista con el movimiento de un vehículo.
		\begin{small}
		\begin{enumerate}
			\item Cambiamos el caso en el que la nave nodriza se movía libremente a aquel en el que se mueve sobre una poligonal cerrada, lo que hemos denominado como \textbf{The Network Mothership-Drone Routing Problem on a Polygonal with Graphs} (PMDRPG). 
			\begin{center}
				\includegraphics[width=0.5\linewidth]{TDMTZ}
			\end{center}
		\end{enumerate}
	\end{small}
	\end{frame}

	\begin{frame}
		\frametitle{Limitación del AMDRPG}
		La principal limitación que tiene este problema es el hecho de que los puntos de lanzamiento y recogida están localizados libremente, algo que no es realista con el movimiento de un vehículo.
		\begin{small}
			\begin{enumerate}
				\setcounter{enumi}{1}
				\item Dimos un paso más allá, presentando una formulación en la que el camión se mueve sobre un grafo, cuyo nombre es \textbf{The Network Mothership-Drone Routing Problem with Graphs} (NMDRPG).
				\begin{center}
					\includegraphics[width=0.5\linewidth]{NDMTZ}
				\end{center}
			\end{enumerate}
		\end{small}
	\end{frame}

	\begin{frame}
		\frametitle{¿Qué dificultades presenta este tipo de problemas?}
		\begin{itemize}
			\item Al tener que diseñar dos rutas coordinadas, el problema se vuelve combinatorio \textit{de forma cuadrática}.
			\item Generan una gran cantidad de variables y restricciones con una cantidad de datos relativamente pequeña. Lo que dificulta la búsqueda del óptimo.
		\end{itemize}
		\begin{center}
			\includegraphics[width=0.25\linewidth]{cerebro}
		\end{center}
	\end{frame}

	\begin{frame}
		\frametitle{Un algoritmo matheurístico para este tipo de problemas}
%		Al ser problemas \textit{muy} combinatorios, lo que se suele buscar son algoritmos que proporcionen \textit{buenas soluciones en un tiempo razonable}. 
		La idea de nuestro algoritmo es descomponer el problema en dos subproblemas fijando variables.
		% \begin{block}
		\begin{footnotesize}
			\begin{enumerate}
				\item Buscamos los centroides de los vértices que forman cada grafo. Y generamos un círculo de un radio determinado centrado en dichos centroides.
				\item Resolvemos el Traveling Salesman Problem with Neighborhoods (TSPN) sobre estos círculos para determinar el orden general de recorrido de los grafos.
				\item Una vez fijado este orden, buscamos el punto de lanzamiento y recogida de forma recursiva sobre cada grafo para fijar el orden de visita dentro de cada grafo.
				\item Resolvemos el modelo con todos estos puntos fijados pero dejando libre el orden general de recorrido.
				\item Vemos si este orden coincide con el anterior. Si no, se repite el Paso 4.
			\end{enumerate}
		\end{footnotesize}
		% \end{block}
	\end{frame}

	\begin{frame}
		\frametitle{Un algoritmo matheurístico para este tipo de problemas}
			\begin{figure}%
				\hspace{-2.5cm}
		    \includegraphics[width=7cm]{step0}\hspace{-1.5cm}\includegraphics[width=7cm]{step1}%
		    \caption{Ejemplo ilustrativo}%
		    \label{fig:example}%
			\end{figure}
	\end{frame}

	\section{Nuevos horizontes}
	\begin{frame}
		\frametitle{Nuevos horizontes}
		Una vez formulados estos problemas, queremos profundizar en este tema en un siguiente documento que pueda responder a las siguientes preguntas en orden de dificultad:
		\begin{itemize}
			\item ¿Qué ocurre si, en vez de disponer solo un dron, tenemos varios drones que podemos lanzar desde un mismo punto y recogerlos a todos en otro punto?
			\pause
			\item ¿Qué ocurre si el dron tiene la suficiente capacidad para visitar más de un elemento antes de regresar a la nave nodriza?
			\pause
			\item ¿Es posible formular el problema en el caso en el que el movimiento de la nave nodriza y/o el dron presente barreras que no puede atravesar?
		\end{itemize}
	\end{frame}

%	\begin{frame}
%		\frametitle{Nuevos horizontes}
%		\begin{center}
%			\includegraphics[width=0.5\linewidth]{continuara}
%		\end{center}
%	\end{frame}

	\begin{frame}
		\vspace{1.5cm}
		\hspace{1.5cm}
		\includegraphics[width=0.7\linewidth]{gracias}
	\end{frame}


\end{document}
